\documentclass[a4]{article}
\usepackage{amsmath}
\usepackage{amssymb}
\usepackage{listings}
\usepackage{graphicx}
\usepackage{../../manual/manual}
\newcounter{lst}[section]

\renewcommand{\thelst}{\arabic{section}.\arabic{lst}}

\lstset{basicstyle=\sffamily,showstringspaces=false}

\lstdefinestyle{easslisting}{basicstyle=\normalfont\footnotesize\sffamily, mathescape=true, frame=tb,  numbers=right, numberstyle=\footnotesize, stepnumber=1, numbersep=-5pt, captionpos=b}

\lstdefinestyle{eass}{basicstyle=\sffamily, mathescape=true}

\lstnewenvironment{listing}[3]{
  \noindent        
  \refstepcounter{lst}         
  \label{code:#1}  
\begin{tabular}{p{.97\columnwidth}} \\ \hline  {\normalsize \textbf{Code  fragment \arabic{section}.\arabic{lst}} #2} \\  \end{tabular} 
\lstset{language=#3,          
  basicstyle=\footnotesize\sffamily,
%%%  basicstyle=\footnotesize,    
  xleftmargin=10pt,    
  mathescape=true,    
  frame=tb,    
  numbers=right,    
%%%  numberstyle=\tiny,     
  numberstyle=\footnotesize,     
  stepnumber=1,     
  numbersep=-5pt}}{}

%-- definition for Gwendolen --%

\lstdefinelanguage{Gwendolen}{%
    morekeywords={Plans,Initial,Beliefs,Goals,name,fof-parse,Rules,Belief,Reasoning},
    morecomment=[l]{//},
 literate= {<-}{{$\leftarrow$}{$\:$}}2
           {.B}{{${\cal B}$}}2
           {.G}{{${\cal G}$}}2
           {lnot}{{$\sim$}}2
           {assert_shared}{{$+_{\Sigma}$}}2
           {remove_shared}{{$-_{\Sigma}$}}2
           {(perform)}{}0
}



\makeindex

\lstset{basicstyle=\sffamily}
\author{Louise A. Dennis}

\title{AJPF Tutorial 3 -- Using AJPF to create models for other Model-Checkers}

\begin{document}
\maketitle
This is the third a series of tutorials on the use of the \ajpf\ model checking program.  This tutorial covers the use of \ajpf\ in conjunction with other model-checkers, specifically \spin\ and \prism.  \ajpf\ is used to create a model of the program which is then verified by another tool.  The main purpose of this is to enable model-checking with more expressive logics (as can be done with the \prism\) implementation, but there may also be efficiency gains in outsourcing property checking to another tool

Files for this tutorial can be found in the \texttt{mcapl} distribution in the directory \texttt{src/examples/gwendolen/ajpf\_tutorials/tutorial3}.

This tutorial assumes familiarity with the operation of \ajpf\ as described in \ajpf\ tutorials 1 and 2 and familiarity with the theory of model-checking.  Unlike most tutorials.  The tutorial is not standalone and assumes the user has access to both \spin\ and \prism.

This tutorial explains how to use the tools described in~\cite{dennis15:two}.

\section{Using \ajpf\ with \spin}

\spin{}~\cite{holzmann04spin} is a popular model-checking tool
originally developed by Bell Laboratories in the 1980s.  It has been
in continuous development for over thirty years and is widely used in
both industry and academia
(e.g.,~\cite{havelund00formal,kars96application,kirsch11technical}).
\spin{} uses an input language called \promela{}.  Typically a model
of a program and the property (as a ``never claim'' --- an automaton
describing executions that violate the property) are both provided in
\promela{}, but \spin{} also provides tools to convert formulae
written in LTL into never claims for use with the
model-checker. \spin{} works by automatically generating programs
written in C which carry out the exploration of the model relative to
an LTL property. \spin{}'s use of compiled C code makes it very quick
in terms of execution time, and this is further enhanced through other
techniques such as partial order reduction. The examples in this tutorial were checked using \spin{} version 6.2.3 (24 October 2012).

\spin\ can be downloaded from \url{http://spinroot.com}.

\bibliographystyle{abbrv} %% {plain}
\bibliography{../../manual/manual}


\end{document}
